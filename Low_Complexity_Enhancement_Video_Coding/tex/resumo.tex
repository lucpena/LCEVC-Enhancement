Este trabalho apresenta uma análise qualitativa do padrão de codificação de vídeo \acrshort{LCEVC}, 
padronizado pela \acrshort{MPEG} e desenvolvido pela V-Nova. O \acrshort{LCEVC} atua como uma camada 
de aprimoramento sobre codecs existentes, oferecendo melhorias de qualidade com baixa complexidade 
computacional. Foram realizados testes com diferentes valores de quantização para a camada base 
(\acrshort{AVC} e \acrshort{VVC}) e para os parâmetros da camada de aprimoramento do \acrshort{LCEVC}. 
Os resultados obtidos, analisados a partir da relação entre \acrshort{PSNR} e o Bitrate, indicam que o 
\acrshort{LCEVC} pode apresentar desempenho superior em certos cenários, especialmente quando comparado 
ao uso isolado de codecs convencionais. A implementação dos testes foi realizada por meio de scripts
automatizados em ambientes Linux. Este estudo reforça o potencial do \acrshort{LCEVC} como uma solução 
eficiente para transmissão de vídeo de alta qualidade com baixo custo computacional.
