O \acrfull{LCEVC} atua como uma camada de aprimoramento 
sobre codificadores convencionais, como o \acrfull{AVC} e o \acrfull{VVC}, com foco em eficiência e baixa 
complexidade computacional. 
Este trabalho realiza uma análise de qualidade do padrão \acrshort{LCEVC}, com o objetivo de identificar os
melhores cenários para o \acrshort{LCEVC}, onde ele é capaz de superar a qualidade ou a taxa de bits de vídeos
que utilizam somente os codificadores que ele aprimora. Estes resultados são analisados para descobrir quais 
seriam os parâmetros que mais têm a chance de gerar uma codificação eficiente para o \acrshort{LCEVC}.
Foram conduzidos experimentos variando os parâmetros de quantização da camada base 
(\acrshort{QP}) e da camada de aprimoramento (\textit{SW2}), utilizando métricas como \acrfull{PSNR} e taxa de 
bits. Os resultados, analisados com base na Fronteira de Pareto, indicam que o \acrshort{LCEVC} apresenta 
desempenho superior em diversos cenários, principalmente em vídeos com resoluções \textit{Full HD} e 4K. 
Além disso, foram identificadas combinações de \acrshort{QP} e \textit{SW2} que resultam em codificações 
mais eficientes, contribuindo para configurações ideais de uso do \acrshort{LCEVC} em aplicações de vídeo 
de alta qualidade com menor custo computacional.
