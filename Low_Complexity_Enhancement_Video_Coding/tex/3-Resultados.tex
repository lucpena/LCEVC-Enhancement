Após a execução do Script de codificação, os dados obtidos são armazenado
em um arquivo CSV. Com os resultados finais, outro script em \textit{Python} é
executado, onde ele cria um gráfico com estes dados. Abaixo estão os resultados
obtidos.

A análise dos gráficos gerados para as sequências testadas permite observar o comportamento 
do \acrshort{LCEVC} em diferentes condições de codificação. Foram analisadas dois
codificadores: \acrshort{AVC} e \acrshort{VVC}, com e sem o uso do \acrshort{LCEVC} 
como camada de aprimoramento.

No geral, os gráficos \acrshort{PSNR} x \textit{Bitrate} permitem avaliar a eficiência da
compressão, considerando que uma melhor relação é obtida quando se atinge maior qualidade
(\acrshort{PSNR}) com menor taxa de bits (\textit{bitrate}). A seguir, apresenta-se uma
análise detalhada por cenário. Para alguns resultados, se um vídeo em \acrshort{LCEVC} 
resultou em um valor muito alto para o \textit{bitrate}, onde fique distante dos valores
do codificador base, eles foram omitidos para melhor análise e visualização nos gráficos.

Além disso, para que o gráfico pudesse ficar legível, só foi rotulado o valor de SW2 para
o parâmetro de QP37.

\newpage
\section{AVC}

\subsection{Bosphorus}
\begin{figure}[h]
    \centering
    \includegraphics[width=1.0\textwidth]{img/Bosphorus-AVC.png}
    \caption{Resultados para "Bosphorus"\ em \acrshort{AVC}. \cite{uvg_dataset}}
    \label{fig:bosphorus}
\end{figure}

Para a sequência Bosphorus, os resultados revelam que o uso do \acrshort{LCEVC} pode
ser vantajoso em determinados cenários, dependendo dos parâmetros utilizados.

Nos testes com valores baixos de SW2, observou-se que, embora o \acrshort{PSNR} 
alcançado seja alto, os arquivos resultantes apresentam um \textit{bitrate}
superior ao necessário para atingir uma qualidade semelhante utilizando
somente o \acrshort{AVC}. 

Para valores intermediários de SW2, os resultados foram mais promissores. Nestes
casos, o \acrshort{LCEVC} foi capaz de alcançar uma boa relação entre qualidade
e taxa de bits. Nestes casos, embora o \acrshort{PSNR} seja um pouco inferior,
o \acrshort{LCEVC} pode ser interessante em contextos onde a reconstrução visual
seja mais importante que o valor do \acrshort{PSNR}.

Para valores mais altos de SW2, como 2250 por exemplo, a camada de aprimoramento
passou a contribuir muito pouco para a codificação, representando uma proporção
bem pequena do tamanho total do arquivo. Neste caso, o \acrshort{LCEVC} possui
sua codificação comprometida pela perda de qualidade e espaço, tornando estas
configurações pouco vantajosas.

\subsection{ReadySteadyGo}

\begin{figure}[h]
    \centering
    \includegraphics[width=1.0\textwidth]{img/ReadySteadyGo-AVC.png}
    \caption{Resultados para "ReadySteadyGo"\ em \acrshort{AVC}. \cite{uvg_dataset}}
    \label{fig:RSG}
\end{figure}

Em geral, os resultados para esta sequência não demonstraram superioridade clara
em comparação com o \acrshort{AVC} puro.

Nos testes com o SW2 = 1250, os resultados demonstram \textit{bitrates} 
significativamente mais altos, com ganhos pequenos ou mesmo nulos em \acrshort{PSNR}.
Mesmo em configurações com um QP menor, o \acrshort{LCEVC} demonstrou um \textit{bitrate}
muito alto, sem nenhuma melhoria de qualidade.

Para valores intermediários de SW2, houve uma redução no \textit{bitrate}, mas
também houve uma queda no \acrshort{PSNR}.

Em valores maiores de SW2, a camada de aprimoramento passou a ter pouca ou nenhuma
relevância, com taxas abaixo dos 10\%. Com isso, os resultados se aproximam do
desempenho da camada base, com valores baixos de \acrshort{PSNR}, acarretando
em uma perda significativa de qualidade.

Desta forma, esta sequência não se beneficiou com o uso do \acrshort{LCEVC}, onde
a relação entre \acrshort{PSNR} e \textit{bitrate} foi mais favorável ao \acrshort{AVC}
puro.

\newpage
\subsection{Jockey}

\begin{figure}[h!]
    \centering
    \includegraphics[width=1.0\textwidth]{img/Jockey-AVC.png}
    \caption{Resultados para "Jockey"\ em \acrshort{AVC}. \cite{uvg_dataset}}
    \label{fig:Jockey}
\end{figure}

Para a sequência \textit{Jockey}, o \acrshort{LCEVC} alcançou ótimos resultados
em relação às sequências de referência, onde há casos em que o \acrshort{LCEVC}
demonstrou um \acrshort{PSNR} superior com um \textit{bitrate} menor. Percebe-se
este comportamento em vários pontos antes do QP27 das sequências de referência.
A partir desse ponto, o valores testados para o \acrshort{LCEVC} decaem e não
são mais os melhores valores. Entretanto, considerando os resultados obtidos,
é possível que com mais testes, surjam mais valores em que o \acrshort{LCEVC}
se sai melhor.

Aqui, vemos um caso em que o \acrshort{LCEVC} se destacou e demonstrou vantagens
em comparação ao \acrshort{AVC} exclusivo, especialmente em configurações de 
\textit{bitrate} intermediários.

Em valores mais baixos, o \acrshort{LCEVC} alcançou \acrshort{PSNR} elevados
com \textit{bitrates} competitivos. Apesar do \textit{bitrate} mais alto,
a relação qualidade-tamanho mostra que o \acrshort{LCEVC} pode ser viável
para aplicações que priorizam a qualidade visual.

Na faixa média, o \acrshort{LCEVC} equilibrou melhor eficiência e qualidade,
superando o \acrshort{AVC} puro. 

Para valores mais altos de SW2, a contribuição da camada de aprimoramento
diminuiu, aproximando-se dos valores de somente \acrshort{AVC}, mas ainda
com ganhos moderados de \acrshort{PSNR}.

Dessa maneira, o \acrshort{LCEVC} é particularmente eficaz para conteúdos
dinâmicos como "Jockey", onde a camada de aprimoramento compensa perdas da
compressão base sem aumentas excessivamente o \textit{bitrate}.

\subsection{SOCCER}

\begin{figure}[h]
    \centering
    \includegraphics[width=1.0\textwidth]{img/SOCCER-AVC.png}
    \caption{Resultados para "SOCCER"\ em \acrshort{AVC}. \cite{xiph}}
    \label{fig:SOCCER}
\end{figure}

Os resultados indicam que o \acrshort{LCEVC} não apresentou vantagem clara em relação
a somente o \acrshort{AVC} na maioria das configurações testadas.

Em valores mais baixo, o \acrshort{LCEVC} gerou arquivos com \textit{bitrates} altos
para pouco ganho em \acrshort{PSNR}, enquanto o \acrshort{AVC} puro atingiu qualidade
similar com taxas de bits muito menores.

Nas faixas intermediárias de SW2, o \acrshort{LCEVC} reduziu o \textit{bitrate}, mas
com perda acentuada de \acrshort{PSNR}, ficando abaixo da curva de eficiência de somente
o \acrshort{AVC}.

Já em SW2 com valores maiores, a camada de aprimoramento se tornou irrelevante, resultando em
desempenho próximo ao da camada base \acrshort{AVC}, porém ainda inferior ao \acrshort{AVC} puro
em termos de \acrshort{PSNR}. 

Assim, o \acrshort{LCEVC} não demonstrou um benefício em ser utilizado nesta sequência.

\newpage
\subsection{City}

\begin{figure}[h]
    \centering
    \includegraphics[width=1.0\textwidth]{img/City-AVC.png}
    \caption{Resultados para "City"\ em \acrshort{AVC}. \cite{xiph}}
    \label{fig:City}
\end{figure}

Para esta sequência, os valores intermediários de SW2 proporcionam os resultados mais equilibrados.
Nesses casos, o \acrshort{LCEVC} foi capaz de gerar reconstruções visuais com ganhos notáveis
em qualidade em relação ao \acrshort{AVC}, porém com um ganho notável em \textit{bitrate}.

Percebe-se que para valores iniciais de SW2, que é 250, os valores
obtidos estão aproximadamente na curva que representa os melhores valores, porém, considerando
o valor para \acrshort{AVC} com QP37, ele possui o mesmo \textit{bitrate} de todos os pontos
iniciais do \acrshort{LCEVC}, porém com um \acrshort{PSNR} superior.

Para valores maiores de SW2, o \acrshort{LCEVC} demonstrou um ganho de qualidade, mas foi
um ganho que não acompanhou o ganho em qualidade do \acrshort{AVC} puro, além do aumento
do tamanho da camada de aprimoramento, que acarretou no aumentos do \textit{bitrate}.

\newpage
\subsection{vc-globo-05}

\begin{figure}[h]
    \centering
    \includegraphics[width=1.0\textwidth]{img/vc-globo-05_120frames-AVC.png}
    \caption{Resultados para "vc-globo-05"\ em \acrshort{AVC}.}
    \label{fig:vc-globo-05}
\end{figure}

Estes resultados mostram que para esta sequência, os resultados obtidos com \acrshort{LCEVC}
foram acima da média em comparação com outros resultados. A curva que resulta dos valores
de SW2 para cada QP com o \acrshort{LCEVC} está mais elevada, demonstrando que houve uma perda
menor de \acrshort{PSNR} e que a qualidade está próxima dos vídeos com somente \acrshort{AVC}.

Os valores 3250 e 2750 de SW2 resultaram em valores de \acrshort{PSNR} que estão acima da curva
dos vídeos de referência, demonstrando mais uma vez que o \acrshort{LCEVC} consegue obter
uma qualidade relativamente melhor utilizando um \textit{bitrate} reduzido.

\newpage

\subsection{vc-lcevc-01}

\begin{figure}[h]
    \centering
    \includegraphics[width=1.0\textwidth]{img/vc-lcevc-01_120frames-AVC.png}
    \caption{Resultados para "vc-lcevc-01"\ em \acrshort{AVC}.}
    \label{fig:vc-lcevc-01}
\end{figure}

Aqui, com o SW2 = 1250, os dados mostraram que a camada de aprimoramento foi fortemente
utilizada, com uma proporção de aprimoramento superior a 90\% em todos os casos, resultando
em um \textit{bitrate} elevado. Reduzindo para o valor 1750, houve uma melhora na eficiência.

Nos valores para SW2 iguais a 1750 e 2250, os resultados foram se aproximando da curva dos
vídeos referência.

Agora, para SW2 com valores de 2750 e 3250, houve uma redução no valor do \textit{bitrate}.
Nestes valores, os resultados foram próximos e até superior aos valores de \acrshort{PSNR}
dos vídeos somente em \acrshort{AVC}.

\newpage

\subsection{vc-philips-01}

\begin{figure}[h]
    \centering
    \includegraphics[width=1.0\textwidth]{img/vc-philips-01_120frames-AVC.png}
    \caption{Resultados para "vc-philips-01"\ em \acrshort{AVC}.}
    \label{fig:vc-philips-01}
\end{figure}

A sequência "vc-philips-01" apresentou uma característica interessante: mesmo
com QPs elevados, os valores de \acrshort{PSNR} se mantiveram altos, indicando
que o conteúdo possui um baixo nível de ruído e variação espacial, o que
favorece a compressão.

Nos testes com SW2 = 750, a proporção da camada de aprimoramento variou entre
84\% e 97\%. Para QP = 30, o \acrshort{LCEVC} atingiu 42,00 db a 15,9 Mbps,
enquanto o \acrshort{AVC} alcançou 43,51 db com apenas 7,5 Mbps. Para valores 
como 1250 e 1750 para o SW2, o \textit{bitrate} reduziu drasticamente
para a maioria dos QPs, mantendo o \acrshort{PSNR} alto.

Esta sequência foi bastante interessante, pois para valores de QPs iguais
a 27, 25 e 22 para o \acrshort{LCEVC}, os resultados foram eficientes,
com um \textit{bitrate} semelhante ao do \acrshort{AVC}, onde o \acrshort{LCEVC}
consegue demonstrar superioridade em alguns casos. 

\newpage

\subsection{vc-philips-03}

\begin{figure}[h]
    \centering
    \includegraphics[width=1.0\textwidth]{img/vc-philips-03_120frames-AVC.png}
    \caption{Resultados para "vc-philips-03" em \acrshort{AVC}.}
    \label{fig:vc-philips-03}
\end{figure}

Os resultados obtidos para esta sequência demonstraram um cenário promissor para
o uso do \acrshort{LCEVC}. Com alguns casos em que ele superou o \acrshort{AVC} puro.
Houve um grande aumento do \textit{bitrate} para os resultados entre os valores de SW2
750 e 1250, onde o aumento foi proporcional ao valor do QP da base.

Vale destacar um dos resultados obtidos pela configuração de SW2 = 1250 e QP = 22, onde
o \acrshort{LCEVC} alcançou um \acrshort{PSNR} de 43,72 db com um \textit{bitrate} 
de apenas 4,1 Mbps, enquanto o \acrshort{AVC} alcançou 47,59 db a 10,5 Mbps. Mesmo
com o \acrshort{PSNR} do \acrshort{AVC} sendo superior, a diferença de quase 2,5 vezes
no \textit{bitrate} pode justificar o uso do \acrshort{LCEVC} em cenários com limitação
de banda.

Em posições intermediárias de SW2, como 1750 e 2250, o \textit{bitrate}, o \acrshort{LCEVC}
se aproximou da curva de eficiência do \acrshort{AVC}, para QPs mais baixos, onde para o QP = 22,
por exemplo, esteve bem próximo do \textit{bitrate} do \acrshort{AVC}.

Para o SW2 = 750, o \textit{bitrate} foi muito alto, onde a camada de aprimoramento
consumiu boa parte do tamanho do arquivo, acarretando em um \textit{bitrate} elevado para um
\acrshort{PSNR} relativamente baixo, que não escalou bem com o aumento do QP.

\newpage

\section{VVC}

\subsection{Bosphorus}

\begin{figure}[h]
    \centering
    \includegraphics[width=1.0\textwidth]{img/Bosphorus-VVC.png}
    \caption{Resultados para "Bosphorus"\ em \acrshort{VVC}. \cite{uvg_dataset}}
    \label{fig:Bosphorus-VVC}
\end{figure}

Com valores baixos de SW2, a camada de aprimoramento possui peso expressivo no tamanho final
do arquivo, acarretando em um \textit{bitrate} elevado. Embora o ganho de qualidade seja pequeno, 
o aumento de \textit{bitrate} é significativo, tornando estes resultados poucos eficientes.

Para valores intermediários, os valores começam a ter um equilíbrio melhor, se aproximando dos
valores de somente \acrshort{VVC}.

Agora, para valores de SW2 altos, como 3250 e 2750, começam a aparecer resultados que estão
acima da curva gerada dos vídeos que utilizaram somente o \acrshort{VVC}. Isso demonstra
que para esta sequência, é possível utilizar o \acrshort{LCEVC} e conseguir resultados
praticamente iguais aos do \acrshort{VVC}.

\newpage
\subsection{SOCCER}

\begin{figure}[h]
    \centering
    \includegraphics[width=1.0\textwidth]{img/SOCCER-VVC.png}
    \caption{Resultados para "SOCCER"\ em \acrshort{VVC}. \cite{xiph}}
    \label{fig:Soccer-VVC}
\end{figure}

Os resultados demonstram que neste caso houve uma vantagem para o uso de
somente o \acrshort{VVC}, com desempenho superior na maioria das configurações
avaliadas, principalmente em eficiência na relação qualidade e taxa.

Aqui a eficiência do \acrshort{LCEVC} foi abaixo do esperado, onde manteve
uma relação alta muita das vezes, indicando baixa eficiência na codificação
da camada base.

Neste caso, o \acrshort{LCEVC} não apresentou vantagens, e muitas vezes
resultou no que seria uma perda de banda, sem ganhos significativos no
\acrshort{PSNR}.

\newpage
\subsection{Jockey}

\begin{figure}[h]
    \centering
    \includegraphics[width=1.0\textwidth]{img/Jockey-VVC.png}
    \caption{Resultados para "Jockey"\ em \acrshort{VVC}. \cite{uvg_dataset}}
    \label{fig:Jockey-VVC}
\end{figure}

A sequência "Jockey" apresentou resultados bastante equilibrados entre o uso do 
\acrshort{VVC} puro e o uso combinado do \acrshort{LCEVC}. Avaliando os  resultados
obtidos, observa-se que o desempenho do \acrshort{LCEVC} e \textit{bitrate} revela
que o desempenho do \acrshort{LCEVC}, neste caso, se aproxima consideravelmente do
\acrshort{VVC} isolado, e até melhor em alguns casos.

Nesta sequência, a qualidade do \acrshort{LCEVC} se manteve consistente e próxima
dos valores obtidos por somente o uso do \acrshort{VVC}. Isso torna o uso do 
\acrshort{LCEVC} nesta sequência uma opção válida e demonstra ser uma alternativa
válida para casos de uso com vídeos semelhantes.

\newpage
\subsection{City}

\begin{figure}[h]
    \centering
    \includegraphics[width=1.0\textwidth]{img/City-VVC.png}
    \caption{Resultados para "City"\ em \acrshort{VVC}. \cite{xiph}}
    \label{fig:City-VVC}
\end{figure}

Na comparação \acrshort{VVC} puro e \acrshort{LCEVC} + \acrshort{VVC}, o desempenho
do \acrshort{VVC} isolado se mostrou eficiente. Esta sequência demonstra que o uso do
\acrshort{VVC} puro se mantém mais eficiente em praticamente todos os cenários testados.

O \acrshort{LCEVC} não trouxe ganhos práticos para esta sequência, onde somente o
\acrshort{VVC} por si só foi mais eficiente e apropriado para codificação de sequências
como este vídeo aéreo de Nova York.

\newpage

\subsection{vc-philips-01}

\begin{figure}[h]
    \centering
    \includegraphics[width=1.0\textwidth]{img/vc-lcevc-01_120frames-VVC.png}
    \caption{Resultados para "vc-philips-01"\ em \acrshort{VVC}.}
    \label{fig:vc-philips-01-VVC}
\end{figure}

Sample Text

\newpage

\subsection{vc-globo-05}

\begin{figure}[h]
    \centering
    \includegraphics[width=1.0\textwidth]{img/vc-globo-05_120frames-VVC.png}
    \caption{Resultados para "vc-globo-05"\ em \acrshort{VVC}.}
    \label{fig:vc-globo-05-VVC}
\end{figure}

Sample Text

\newpage

\section{Considerações}

\begin{itemize}
    \item O comportamento do \acrshort{LCEVC} varia conforme a sequência testada, o
    que é esperado, já que o padrão atua como uma camada adaptativa;

    \item O \acrshort{LCEVC} demonstrou um desempenho superior em sequências que
    envolvam mais movimentação de câmera e detalhes, como no caso da sequência "Jockey";
    
    \item A vantagem do \acrshort{LCEVC}  se mostra mais evidente em cenários de
    \textit{bitrate} mais restrito, onde o refinamento da imagem se torna crucial;


    \item Em taxas mais altas, o benefício da camada de aprimoramento tende a diminuir,
    pois a camada base já está oferecendo uma boa qualidade;

    \item Também é importante observar que o uso do \acrshort{LCEVC} com codecs mais
    simples, como o \acrshort{AVC} tende a ser mais vantajoso do que com outros codecs
    mais avançados, como o \acrshort{VVC}, uma vez que o ganho sobre algo já bem
    otimizado costuma ser menor;

    \item Nos testes realizados, foi possível observar que mesmo quando há uma proporção
    bem alta para a camada de aprimoramento, o ganho de \acrshort{PSNR} não é direto, onde
    muitos casos o \textit{bitrate} do \acrshort{LCEVC} é muito maior que um vídeo utilizando
    somente o codificador base, e a qualidade aferida pelo \acrshort{PSNR} é muito menor.
\end{itemize}

\newpage
\section{Resumo dos resultados}

As tabelas demonstram um breve resumo da conclusão de cada gráfico para uma análise 
geral.

\begin{table}[h]
    \centering
    \begin{tabular}{|c|c|}
        \hline
        \textbf{Sequência} & \textbf{Resultado}\\
        \hline
        Bosphorus (1920x1080, AVC) & Leve vantagem para o LCEVC\\
        \hline
        ReadySteadyGo (1920x1080, AVC) & Vantagem para o AVC\\
        \hline
        Jockey (1920x1080, AVC) & Vantagem para o LCEVC\\
        \hline
        SOCCER (352x288, AVC) & Vantagem para o AVC\\
        \hline
        City (704x576, AVC) & Vantagem para o AVC\\
        \hline
        vc-globo-05 (3840x2160, AVC) & Empate\\
        \hline
        vc-lcevc-01 (3840x2160, AVC) & Empate\\
        \hline
        vc-philips-01 (3840x2160, AVC) & Empate\\
        \hline
        vc-philips-03 (3840x2160, AVC) & Leve vantagem para o LCEVC\\
        \hline
    \end{tabular}
    \caption{Compilado dos resultados finais para AVC.}
    \label{tab:results-avc}
\end{table}

\begin{table}[h]
    \centering
    \begin{tabular}{|c|c|}
        \hline
        \textbf{Sequência} & \textbf{Resultado}\\
        \hline
        Bosphorus (1920x1080, VVC) & Empate\\
        \hline
        SOCCER (352x288, VVC) & Vantagem para o VVC\\
        \hline
        Jockey (1920x1080, VVC) & Empate\\
        \hline
        City (704x576, VVC) & Empate\\
        \hline
        vc-globo-05 (3840x2160, VVC) & TBA\\
        \hline
        vc-lcevc-01 (3840x2160, VVC) & Empate\\
        \hline
    \end{tabular}
    \caption{Compilado dos resultados finais do VVC.}
    \label{tab:results-vvc}
\end{table}