Todos os códigos criados para este trabalho estão disponíveis no GitHub \cite{my_scripts}.
Recomenda-se o GitHub para a análise do código, devido a organização dos arquivos.
Para garantir a preservação destes códigos, eles serão anexados abaixo.

Os scripts foram divididos em duas baterias de codificação diferentes. Uma para 
\acrshort{AVC} e uma para \acrshort{VVC}. Cada um possui seu \textit{main} que
irá chamar todos os códigos necessários de maneira paralela, para agilizar a 
geração dos resultados. Entretanto, este script foi executado em um processador
com vários núcleos que se beneficia com a paralelização. Caso for executar este
script em um processador com menos recursos, recomenda-se a alteração do script
para que ele seja executado sequencialmente, ou diminuir a quantidade de processos
criados.

O script cria uma pasta chamada \textit{avc\_only}, onde é necessário colar nesta
pasta o \textit{encoder.cfg} do codificar \acrshort{AVC}. Também é possível alterar
o script e passar o endereço do \textit{encoder.cfg}.

Coloque todos estes arquivos na mesma pasta.

% \subsubsection{}
% \begin{lstlisting}

% \end{lstlisting}

\section{AVC}

\subsubsection{main-avc.sh}
\lstinputlisting{code/main-avc.sh}

\subsubsection{avc\_only\_template.sh}
\lstinputlisting{code/avc_only_template.sh}

\subsubsection{avc\_template.sh}
\lstinputlisting{code/avc_template.sh}

\section{VVC}
\subsubsection{main-vvc.sh}
\lstinputlisting{code/main-vvc.sh}

\subsubsection{vvc\_only\_template.sh}
\lstinputlisting{code/vvc_only_template.sh}

\subsubsection{vvc\_template.sh}
\lstinputlisting{code/vvc_template.sh}

\section{Resample}

O script para \textit{Resample} é o mesmo para os dois \textit{mains}.
O resultado do \textit{Resample} foi desconsiderado por não contribuir de maneira
útil para o resultado, mas ainda é necessário para que o script funcione.

\subsubsection{resample\_template.sh}
\lstinputlisting{code/resample_template.sh}