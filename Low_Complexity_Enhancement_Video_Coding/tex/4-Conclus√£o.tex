Este trabalho apresentou uma análise qualitativa do padrão \acrfull{LCEVC},
investigando seu desempenho quando utilizado como camada de aprimoramento sobre
codecs tradicionais, como \acrshort{AVC} e \acrshort{VVC}. A proposta do 
\acrshort{LCEVC} de oferecer melhorias de qualidade com baixa complexidade
é bastante útil para o contexto atual dos codificadores de vídeo.

Os teste realizados evidenciaram que o desempenho do \acrshort{LCEVC} é 
fortemente atrelado ao tipo de conteúdo codificado e à parametrização
aplicada, especialmente os valores de QP para a camada base e o SW2
para a camada de aprimoramento. De maneira geral, o \acrshort{LCEVC}
demonstrou resultados positivos em algumas condições específicas,
particularmente em conteúdos com maior movimentação e complexidade
temporal, como a sequência "Jockey", onde a camada de aprimoramento
se mostrou eficaz na preservação da qualidade virtual, com uma taxa de bits
eficiente.

Em outras sequências como "Soccer" e "City" os resultados indicam que
o isolado dos codecs foi mais eficiente. O uso do \acrshort{LCEVC} com
uma baixa quantização (SW2) resultou em taxas de bits elevadas, sem ganhos
de \acrshort{PSNR}.

Os resultados indicam que a principal vantagem do \acrshort{LCEVC} está
na sua flexibilidade de adaptação ao cenário. Quando ele é bem parametrizado,
ele oferece uma relação qualidade e taxa de bits competitiva, tornando-o uma
solução atrativa para ambientes com restrições de hardware, banda ou complexidade
computacional, onde ele permite que seja alcançado níveis satisfatórios de qualidade
mesmo quando não é viável utilizar codecs mais exigentes. Outro ponto forte é sua
alta customização de parâmetros, que o torna uma tecnologia com alto nível de
adaptação a vários cenários. 

Porém, seu uso requer cuidado nas escolhas dos parâmetros, porém, seus ganhos em eficiência nem sempre
são garantidos, principalmente quando se utiliza codecs modernos, como observado
com o \acrshort{VVC}.

\section{Trabalhos Futuros}

Embora este trabalho tenha explorado o \acrshort{LCEVC} em combinação com codecs
tradicionais, há várias direções futuras que podem ser investigadas:

\begin{itemize}
    \item Explorar o uso do \acrshort{LCEVC} com outros codecs, como \acrshort{HEVC} e \acrshort{EVC}, para avaliar
    seu desempenho em diferentes cenários de compressão.
    \item Investigar a aplicação do \acrshort{LCEVC} em contextos de transmissão ao vivo, onde a latência e a eficiência
    são críticas.
    \item Analisar o impacto do \acrshort{LCEVC} em dispositivos com recursos limitados, como smartphones e dispositivos
    embarcados, para entender melhor sua viabilidade em cenários de baixa potência.
    \item Realizar uma análise mais aprofundada da complexidade computacional do \acrshort{LCEVC}, comparando-a com outros
    métodos de aprimoramento de vídeo.
    \item Investigar a integração do \acrshort{LCEVC} com técnicas de aprendizado de máquina para otimização
    de parâmetros e melhoria da qualidade do vídeo.
    \item Realizar um estudo mais profundo sobre os parâmetros de codificação do \acrshort{LCEVC}, como o SW1 e outras centenas
    de parâmetros disponíveis, para entender melhor como eles afetam a qualidade e a eficiência da codificação.
    \item Explorar a possibilidade de utilizar o \acrshort{LCEVC} em conjunto com técnicas de super-resolução
    para melhorar ainda mais a qualidade do vídeo em resoluções mais baixas. Como por exemplo, o uso do \acrfull{DLSS} ao
    invés do algoritmo de \textit{Upsampling} do \acrshort{LCEVC}.
\end{itemize}