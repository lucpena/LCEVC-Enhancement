Este trabalho apresentou uma análise de qualidade do \acrshort{LCEVC}, comparando seus resultados
com vídeos que só utilizaram o codificar base utilizado. O \acrshort{LCEVC} demonstrou um resultado
promissor, onde vários pontos foram superiores aos resultados obtidos pelo codificador da camada base.
Com a análise dos resultados obtidos, foi possível perceber que o \acrshort{LCEVC} realmente proporciona
benefícios de qualidade e tamanho, e com uma complexidade reduzida.

Os resultados mostram que o \acrshort{LCEVC} se sai bem com vídeos de resoluções como Full HD (1080p) e 
4k, mas não se saiu bem com vídeos com resoluções menores que isso. Dentro dos resultados onde o 
\acrshort{LCEVC} superou os vídeos somente com o codificador base, alguns valores de configuração
para o \acrshort{LCEVC} apareceram mais, demonstrando serem valores ideias para serem usados.
Estes valores foram: \textbf{SW2 = 3250 com QP = 30}, \textbf{SW2 = 3250 com QP = 27} e \textbf{SW2 = 3250 com QP = 25}
onde apareceram 10 vezes cada em pontos acima da curva de eficiência. Outros valores que também
demonstraram eficiência foram \textbf{SW2 = 2750 com QP = 27} e \textbf{SW2 = 2750 com QP = 25}. Estes valores
apareceram 9 e 8 vezes respectivamente. Esses pares de valores demonstraram ser os melhores candidatos a 
uma codificação que resulte em um resultado melhor que uma codificação utilizando somente o codificador
da camada base. 

Com isso, o \acrlong{LCEVC} mostrou ser capaz de ser mais eficiente que vídeos utilizando somente a camada
base, e que é uma opção válida a ser usada junto dos codificadores atuais. Com estes resultados positivos,
o \acrshort{LCEVC} se mostra uma ótima tecnologia a ser incorporada em aplicações como a TV 3.0, 
onde o projeto original já demonstra sua viabilidade \cite{tv_25, tv_30}.

\section{Trabalhos Futuros}

Embora este trabalho tenha explorado o \acrshort{LCEVC} em combinação com codecs
tradicionais, há várias direções futuras que podem ser investigadas:

\begin{itemize}
    \item Explorar o uso do \acrshort{LCEVC} com outros codecs, como \acrshort{HEVC} e \acrshort{EVC}, para avaliar
    seu desempenho em diferentes cenários de compressão.
    \item Investigar a aplicação do \acrshort{LCEVC} em contextos de transmissão ao vivo, onde a latência e a eficiência
    são críticas.
    \item Analisar o impacto do \acrshort{LCEVC} em dispositivos com recursos limitados, como smartphones e dispositivos
    embarcados, para entender melhor sua viabilidade em cenários de baixa potência.
    \item Realizar uma análise mais aprofundada da complexidade computacional do \acrshort{LCEVC}, comparando-a com outros
    métodos de aprimoramento de vídeo.
    \item Investigar a integração do \acrshort{LCEVC} com técnicas de aprendizado de máquina para otimização
    de parâmetros e melhoria da qualidade do vídeo.
    \item Realizar um estudo mais profundo sobre os parâmetros de codificação do \acrshort{LCEVC}, como o SW1 e outras centenas
    de parâmetros disponíveis, para entender melhor como eles afetam a qualidade e a eficiência da codificação.
    \item Explorar a possibilidade de utilizar o \acrshort{LCEVC} em conjunto com técnicas de super-resolução
    para melhorar ainda mais a qualidade do vídeo em resoluções mais baixas. Como por exemplo, o uso do \acrfull{DLSS} ao
    invés do algoritmo de \textit{Upsampling} do \acrshort{LCEVC}.
\end{itemize}