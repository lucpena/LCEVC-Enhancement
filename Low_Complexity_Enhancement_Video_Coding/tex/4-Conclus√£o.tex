Este trabalho apresentou um a análise qualitativa do padrão \acrfull{LCEVC},
investigando seu desempenho quando utilizado como camada de aprimoramento sobre
codecs tradicionais, como \acrshort{AVC} e \acrshort{VVC}. A proposta do 
\acrshort{LCEVC} de oferecer melhorias de qualidade com baixa complexidade
é bastante útil para o contexto atual dos codificadores de vídeo.

Os teste realizados evidenciaram que o desempenho do \acrshort{LCEVC} é 
fortemente atrelado ao tipo de conteúdo codificado e à parametrização
aplicada, especialmente os valores de QP para a camada base e o SW2
para a camada de aprimoramento. De maneira geral, o \acrshort{LCEVC}
demonstrou resultados positivos em algumas condições específicas,
particularmente em conteúdos com maior movimentação e complexidade
temporal, como a sequência "Jockey", onde a camada de aprimoramento
se mostrou eficaz na preservação da qualidade virtual, com \textit{bitrate}
eficiente.

Em outras sequências como "Soccer" e "City" os resultados indicam que
o isolado dos codecs foi mais eficiente. O uso do \acrshort{LCEVC} com
uma baixa quantização (SW2) resultou em taxas de bits elevadas, sem ganhos
de \acrshort{PSNR}.

Os resultados indicam que a principal vantagem do \acrshort{LCEVC} está
na sua flexibilidade de adaptação ao cenário. Quando ele é bem parametrizado,
ele oferece uma relação qualidade e taxa de bits competitiva, tornando-o uma
solução atrativa para ambientes com restrições de hardware, banda ou complexidade
computacional, onde ele permite que seja alcançado níveis satisfatórios de qualidade
mesmo quando não é viável utilizar codecs mais exigentes. Outro ponto forte é sua
alta customização de parâmetros, que o torna uma tecnologia com alto nível de
adaptação a vários cenários. 

Porém, seu uso requer cuidado nas escolhas dos parâmetros, e o ganho em eficiência nem sempre
é garantido, principalmente quando se utiliza codecs modernos, como observado
com o \acrshort{VVC}.